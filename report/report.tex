\documentclass[11pt,a4paper,british,twocolumn]{bhamarticle}
\usepackage[utf8]{inputenc}
\usepackage{multicol}
\usepackage{float}
\usepackage{babel}
\usepackage{csquotes}
\usepackage{isodate}
\usepackage[usenames, dvipsnames]{color,xcolor}
\usepackage{listings}

\usepackage[hidelinks]{hyperref}
\hypersetup{colorlinks, linkcolor={red!50!black}, citecolor={blue!50!black}, urlcolor={blue!80!black}}

% Configure bibliography
\usepackage[backend=biber,sorting=none,citestyle=numeric-comp]{biblatex}
\addbibresource{references.bib}

\usepackage{report}

% chktex-file 36

% Tikz for diagrams
\usepackage{tikz}
\usetikzlibrary{calc,shapes,arrows.meta,positioning}
\tikzstyle{box}   = [rectangle, minimum height=1.5em, minimum width=4em, draw=black!50, thick]
\usepackage{bytefield}

\title{A TCP/IP networking stack implementation in userland}
\author{Joe Groocock -- 1467414} % chktex 8
\supervisor{Dr Ian Batten}
\department{Computer Science}
\degree{MSc in Computer Science}

\begin{document}

\maketitle

\tableofcontents
\newpage
\pagenumbering{arabic}

% TODO: Make abstract relevant to final project
% - it states the problem that you set out to solve
% - it describes your solution and method
% - it states a conclusion about the success of the solution
\begin{abstract}
    TCP/IP network stacks included in most modern operating systems are opaque to the user and mostly unconfigurable. This article aims to design and produce a coherent specification for an efficient, high performance TCP/IP network stack that can be used as a software library, standalone application or system-wide daemon to provide network access to many applications. It is provided as a learning and academic research tool that can be tweaked, extended and analysed in depth to aid in gaining a better understanding of how operating systems handle networking and TCP/IP as whole.
\end{abstract}

%! TEX root=../report.tex

\chapter{Introduction}\label{chp:intro}
% Describe the problem to be solved
The primary objective of this project is to construct a portable TCP/IP implementation running in userland with no reliance on kernel networking. Network operations in most modern operating systems are passed down to the \textit{kernel} for processing, to maximise performance.

    % Explain enough background to understand the problem
    % Clearly state project aims
    \section{Aims}\label{sec:aims}
    Learning about the inner workings of TCP, IP and other protocols with the black-box approach using the kernel implementations is difficult. Producing modifications to interwork with existing code requires compilation of some or all of the kernel in order to be applied, impeding productivity. Many of these drawbacks can be overcome by moving the networking code up into userspace. Requiring only a single passage to transfer complete ethernet packets back and forth to the physical network interface, a network software utility can be entirely self-sufficient and thus much more manageable.

    Providing a software library to perform network protocol operations opens up many opportunities for further research into userland adaptations of existing protocols or prototyping new protocols with greater ease. Access to the inner workings of protocols has potential to allow for new types of network packet inspection or injection, for example a \textit{virtual private network} tunnelling implementation that relies solely on packet-level access for data injection.

    As a learning tool it is important that it performs correctly, as per the specification, and is performant enough to be useful. Many implementations are littered with additional functionality for legacy systems or rarely-used features that lessen the overall learning experience. Code should be well commented and readable whilst still not sacrificing performance.

    \section{Structure}
    % Provide an overview of structure of solution?
    Making up the whole solution to this project are multiple modular components that when combined can be used in many different ways, for example some of the provided use cases in Section~\ref{sec:aims}.
    Most importantly is the network stack implementation which provides an end-to-end process for handling TCP, IP and other protocols. The network stack is comprised of many layers, as discussed in-depth in Chapter~\ref{chp:design}. To integrate seamlessly with existing software, \texttt{libnshook} and \texttt{libnetstack} can be used to replace the default networking in Unix application binaries.

    % Explain what is in each section of this document
    In Chapter~\ref{chp:spec} the project goals are formalised into requirements. Chapter~\ref{chp:background} overviews the history and operation of TCP and IP protocols. Discussion of other works, relating to the topic of this project are covered in Chapter~\ref{chp:related}. Detailed information about how the project was designed and implemented are covered in Chapters~\ref{chp:design}~\&~\ref{chp:impl}. An overview of the testing tools and strategies used is in Chapter~\ref{chp:testing}. The project outcomes and experiences are summarised and evaluated in Chapter~\ref{chp:eval}.


%! TEX root = ../report.tex

\section{Related work}
    \subsection{\mbox{Overview}}
        % RFC675 overview
        When TCP was originally proposed in December 1974, by \citeauthor{rfc675} in RFC 675~\cite{rfc675}, flow diagrams and implementation suggestions were defined but lacked specific pseudo-code routines that could be implemented directly, were not provided. At that time TCP was young and naive due to having little exposure to real-world use cases. Furthermore, the limitations of the technology of the era were apparent in the original design given the scope and proposed use of the protocol, compared to the significantly higher speed demands in 2017.

        % RFC793 overview & level-ip implementation to the spec
        RFC 793 followed some years later and provided a more specific breakdown, that can be directly translated and implemented in code, of each event within the protocol and an appropriate procedure to handle it~\cite[Page~54-77]{rfc793}. \citeauthor{rfc793} who published the specification, now recognised as an official internet standard, defined TCP as `a connection-oriented, end-to-end reliable protocol' as it is now widely known. The document provides a well-defined list of requirements for the protocol to run, as well as a multitude of provisions for successful operation within the promises.
        Many implementations of the TCP/IP stack including, but not limited to, `Level-IP' by \citeauthor{levelip-spec} follows this specification~\cite[\texttt{src/tcp\_input.c} line~262]{levelip-spec} very closely which, in theory, produces a TCP that should interwork seamlessly, as per the specification, with any other correctly implemented TCP. % chktex 13

        % lwIP goals
        Not all TCP/IP stacks are created equal; there are numerous incentives for developing alternate implementations, for example `lwIP' from \citeauthor{lwip}~(\citeyear{lwip}) was built `to reduce memory usage and code size, making lwIP suitable for use in small clients with very limited resources such as embedded systems'. There are many inefficiencies in `standard' networking stacks like those included in popular operating systems such as Linux, BSD, macOS and Windows, to name a few, especially regarding memory usage. These protocols make the assumption that the physical system has considerable amounts of memory available for receiving, processing and duplicating network data both in the kernel and in user applications.

        \citeauthor{lwip} makes the opposite assumption and as a result produced a system where minimal replication of data and little wasted memory allocation occurs. Using dynamically sized packet buffers \texttt{pbufs}, \citeauthor{lwip} made efficient use of RAM, ROM and pooled memory to address network data without requiring it to be copied to a local storage space before being actioned. Through many small optimisations like these, lwIP was, and still is, a very effective network stack, usable on even the most restricted hardware which in the growing interconnected embedded device market is invaluable.
        % TODO: Maybe talk about uIP here, if I have space/time/the will to live

        % mTCP goals
        Conversely, some alternate implementations exist for quite the opposite reasons such as `mTCP' \citeauthor{jeong2014mtcp}, which was constructed as `a highly scalable user-level TCP stack for multicore systems'. The intention was for mTCP to outperform competing solutions in packet throughput and data volume. According to their claims, mTCP can surpass Linux by a factor of 25 in `small message transactions' while also boosting regular performance of popular applications between 33\% and 320\%. Such performance numbers are impressive, especially considering the widespread use of Linux for commercial applications and hosting, which begs the question: ``Why is it significantly faster than the default Linux implementation and why hasn't Linux caught up yet?''~\cite[2.2, 3]{jeong2014mtcp}

        Many of the improvements suggested by \citeauthor{jeong2014mtcp}~\cite{jeong2014mtcp} are very intelligent applications of high-speed network adapters and multicore processor systems, such as servers. Any TCP network stack that is to be run in these kinds of scenarios should consider these optimisations for improved performance. It is likely that many of the proposed enhancements would also benefit low-power single CPU systems too, with the most probable outcome being reduced latency and memory usage.

    \subsection{Demultiplexing TCP}\label{sec:demultiplex}
        \citeauthor{braun:inria-00074040} designed a modified BSD TCP/IP stack where the IP layer resides in the kernel and TCP is split in two between kernel and userspace into TCPU and TCPK respectively. TCP processing is moved mostly into the user region, residing as local code in the calling process. The only exception to this is the `demultiplexing' step, TCPK, where TCP frames are routed to the correct user program based on the ports and addresses from the IP layer packet, providing security for the receiving process.

        % Demultiplexing in userspace, or not
        \citeauthor{braun:inria-00074040} theorised that a potential alternative method for demultiplexing packets would be to pass all incoming traffic directly into a userspace daemon for processing, removing the requirement on kernel modifications. However, it is concluded that this concept is impractical and inefficient compared with the alternative solution (above) as packets are processed by two userspace applications, causing more context switches passing data from the daemon to the receiving processes~\cite[2.1]{braun:inria-00074040}\cite[3]{edwards1995experiences}. Generally, this assumption of relative inefficiency is true, however, in certain circumstances there can be situations where this is actually a practical and viable solution. This project aims to implement a usermode-only networking stack, meaning kernel modifications are not plausible and therefore this solution is ideal when optimised appropriately, reducing the context switching overhead.

    \subsection{Throughput performance}\label{sec:thruput}
        Much of the research surrounding TCP, particularly that providing userspace implementation detail, is focussed on optimising the protocol for high performance and high throughput. \citeauthor{edwards1995experiences}, in \citeyear{edwards1995experiences} demonstrated throughput of 160 Mbit/s using a userspace TCP implementation running over coaxial token-ring ATM networking, making use of the solid existing HP-UX kernel TCP code along with single-copy from the NIC to the user application using shared memory in the TCP stack and the application. This particular arrangement managed 80\% of the performance of the existing single-copy TCP stack and 150\% of the double-copy kernel stack. \citeauthor{braun:inria-00074040} in \citeyear{braun:inria-00074040} yielded similar results of around 40-50\%~\cite[5]{braun:inria-00074040} between the default kernel and TCPU/TCPK stacks running on much less powerful hardware. % chktex 8
        % TODO: Talk about lwIP here (2,6,12)
        %     Copying from kernel <-> user

        % mTCP thread-local storage and multicore
        \citeauthor{jeong2014mtcp} shows that with the use of multiple receive queues spread across individual CPU cores and fewer locks along with improved buffer management and fewer context switches between kernel and user mode their implementation can yield a much improved throughput compared to Linux and other user-mode concepts. Much of the improved performance is thanks to the use of \textit{thread local storage} of socket buffers, TCP buffer pools and other data structures relating to individual threads that are not shared. Greatly reduced usage of locks across threads, reducing \textit{Lock contention}, helps to minimise idle time in both incoming and outgoing packet processing.

    \subsection{\mbox{Reducing data copy overhead}}\label{sec:reducecopy}
        A common theme across many protocol implementations is the focus on reducing overhead due to data copying. Any copied data is potentially wasted time, especially if the same functionality could be programmed using a \textit{zero-copy} approach, where incoming packets are deposited directly from the network adapter into the user program without being duplicated one or more times. Many of the related works include some variation of a transmit/receive queue utilised by every layer from the NIC to the user program, minimal or zero data copying. \citeauthor{jeong2014mtcp} utilised an event-driven packet I/O library to divide incoming batches of packets into multiple queues written directly from the network interface and passed directly through into the user TCP~\cite[3.1]{jeong2014mtcp}.

        \citeauthor{braun:inria-00074040} took a different approach to reducing data copying where typically packet buffers are copied twice: once from the device to an input buffer queue then secondly into the application buffer. In certain situations, by the process of \textit{header prediction}, the kernel copies the header data only. Then assuming the predicted header size was correct, enclosed packet data is copied directly into the user program receive buffer~\cite[4.1]{braun:inria-00074040}. By modern standards this is less than optimal, as now there exists suitable methods to skip copying entirely and use memory-mapped buffers.

        Due to the limitations of the BSD socket API, \textit{zero-copy} networking is not trivially implemented without some significant changes to how data is copied into the user memory, although there is one solution that harnesses the power of hardware offload. Using a feature of virtual memory to remap memory pages, data can be offloaded by the network interface directly into aligned memory pages, separating header from the payload~\cite[2.3]{chase2001end}. Specific hardware and driver support is required for this, and even more complex connection tracking is required for TCP offload. A result of the memory page mapping is large contiguous blocks of memory containing packet payloads produced without a single data copy.

    \subsection{\mbox{TCP security considerations}}
        By moving processing from kernel to userspace, protections such as obscurity from other programs is reduced. \citeauthor{braun:inria-00074040} raised the issue that with a userspace TCP process, data for multiple connections is loaded into the memory space of a single process, enabling it to be viewed or modified by any privileged program~\cite[1, 2.1]{braun:inria-00074040}. In most cases this is not a significant issue, however, it should still be considered. The only plausible solution is to move the demultiplexing stage back into the kernel to restore the connection security.

%! TEX root = ../report.tex

\section{Specification}
    % How you analysed the problem
    % Give an appropriate specification of the solution

    % Requirements from journal
    % \begin{itemize}
    %     \item This project must produce a TCP/IP implementation that runs completely in Linux userspace with no/minimal kernel modification
    %     \item This project must follow the original RFC 793~\cite{rfc793} specification for TCP
    %     \item This project must provide a network stack as a library such that it can be easily implemented in other software
    %         \begin{itemize}
    %             \item There should be sufficient documentation to demonstrate examples of how this can be achieved
    %         \end{itemize}
    %     \item This project should provide sufficient interface to the user to view the inner workings of a TCP connection
    %     \item This project could implement modern TCP extensions such as Window Scaling, Selective \texttt{ACK} and Congestion Control.
    % \end{itemize}

    % Define specific behavior or functions
    % what a software system should do
    \subsection{Functional Requirements}

    % A requirement that specifies criteria that can be used to judge the operation of a system, rather than specific behaviors.
    % constraints on how the system should do so
    % elaborates a performance characteristic
    \subsection{Non-functional Requirements}

    % What the user expects the software to be able to do
    \subsection{User Requirements}
        

%! TEX root = ../report.tex

\section{Design}
    \subsection{Layered Architecture}
        \begin{figure}[H]
            \fontsize{7pt}{7pt}
            \centering
            \begin{tikzpicture}[node distance=1.9em and 2.0em]
                \node [box] (rs) {Raw Socket};
                \node [box, above = of rs] (e) {Ethernet};
                \node [box, above = of e] (ipv4) {IPv4};
                \node [box, above right = of ipv4, xshift=-3em] (tcp) {TCP};
                \node [box, above left = of ipv4,xshift=3em] (icmp) {ICMP};
                \node [box, left = of e] (arp) {ARP};
                \node [box, above = of arp] (r) {Routing};
                \node [box, right = of rs] (tap) {TAP device};
                \node [box, above = of tap] (tun) {TUN device};

                \node [left  = of rs, xshift=-5em] (1) {\parbox[l]{6.5em}{\raggedleft\textcolor{gray}{\textit{Physical (1)}}}};
                \node [above = of 1]               (2) {\parbox[l]{6.5em}{\raggedleft\textcolor{gray}{\textit{Link (2)}}}};
                \node [above = of 2]               (3) {\parbox[l]{6.5em}{\raggedleft\textcolor{gray}{\textit{Internet (3)}}}};
                \node [above = of 3]               (4) {\parbox[l]{6.5em}{\raggedleft\textcolor{gray}{\textit{Transport (4)}}}};

                \draw[-] (rs) -- (e);
                \draw[-] (e.south east) -- (tap.north west);
                \draw[-] (ipv4.south east) -- (tun.north west);
                \draw[-{Stealth[open,scale=1.6]}] (e) -- (ipv4);
                \draw[-{Stealth[open,scale=1.6]}] (ipv4) -- (r);
                \draw[-{Stealth[open,scale=1.6]}] (r.south east) -- (e.north west);
                \draw[-] (arp) -- (r);
                \draw[-] (arp) -- (e);
                \draw[-] (ipv4) -- (tcp.south);
                \draw[-] (ipv4) -- (icmp.south);
            \end{tikzpicture}
            \caption{Protocol layers}\label{fig:layer-arch}
        \end{figure}

        Network protocols are, by design, layered~(\ref{fig:layer-arch}). This design feature can be used as a building block to structure, both semantically and in code, how the program layers will slot together.At the bottom of modern networks is the \emph{link layer}, in almost every case \emph{ethernet}. At this level, whole packets `datagrams' are exchanged between the outside world and the network stack through one or more \emph{interface}s. The method for receiving and sending these packets is unimportant but the interface for doing so must be standardised to allow for interchangeable interfaces at runtime. Depending on the operating system or use-case requirements, it may be preferable to use different interface implementations for the situation. \todo{Ref to interface type choices here}

        Within the ethernet packets can reside many different protocols; the particular one of importance for most internet applications is the \emph{internet protocol}, specifically version 4. It should be noted that the newer and much improved IPv6 should also be accessible here, as a drop-in replacement, or to be run alongside IPv4 at the same time.

        Resolution of 48-bit ethernet hardware addresses, required for routing IPv4 packets to devices on the local network before they can be sent, is performed by the \emph{address resolution protocol} (ARP). This protocol also leverages the lower ethernet packets as transport.

        The ambiguity of payload type within the ethernet packets is resolved with a type field in the header of the ethernet packet with fixed values defined by IEEE~\cite{ieee-ethertype}. Given this information, a choice can be made for each incoming packet to which upper layer protocol the data should be forwarded to. This pluggable design makes implementation simple as well as future compatibility with newer protocols.

        Much like the nested and layered nature of ethernet and inner protocols, the internet protocol is much the same. Two of the vast collection of protocols that run within IP that this software is focusing on are \emph{transmission control protocol} (TCP) and \emph{internet control message protocol} (ICMP)\@. Whilst both having a completely different structure and design, both IPv4 and IPv6 share one common trait: they are interchangeable in the sense that the data they carry is identical and the link layers that carry them are the same. Provided two or more communicating hosts in a connection support one common IP version a link can be established, providing the same internet communication layer guarantees for upper protocols, regardless of IP version used.

    \subsection{TCP buffer management}
        As specified in RFC793~\cite{rfc793}, there is no requirement to retain out-of-order received segments, nor is it required to reconstruct outgoing retransmission segments, but in both cases retaining this extra data can be useful. Typically this optimisation can yield minor improvements in performance at the cost of greater memory usage or additional re-computation.

        Historically, copying data between buffers has been slow so the amount of buffers used should be kept to a minimum. Modern memory copy implementations are highly optimised and the overhead is minimal in comparison to naively copying bytes but still should be considered.

        \subsubsection{Transmission Buffers}
            Caching of outgoing segments before \texttt{ACK}s arrive can negate the requirement for an additional send buffer to hold the unacknowledged data however does also require retaining the potentially unwanted header information attached to the stored packets too. To reduce retransmission overhead, it is common to simply retransmit the original segments instead of reconstructing them from more up-to-date information. Implementations in popular operating systems like Linux and Windows use a hybrid approach where one or more of the original segments is retransmitted initially followed by newly crafted packets upon receipt of an ACK for the initial retransmission(s). \\ % chktex 36
            There are both benefits and drawbacks to this approach, one being that retransmitted segments do not have to be recomputed, reducing overhead. Unmodified segments will likely not contain the latest control information such as window size or \texttt{ACK} value; in this case these packets will be almost indistinguishable to a delayed segment to the peer TCP\@.

            In some cases, partially full packets are retransmitted which can be considered unfulfilled potential as the ratio of control information to payload is lower than optimal when there is extra data in the retransmission queue. Reconstructing or amending smaller segments, incorporating more data to saturate the packet, can improve throughput and reduce latency. One of the major performance-limiting factors of TCP connections, especially those over high-latency links, is waiting for replies or timeouts. When the quantity of packets sent (and as a result, waited on) can be reduced, the delay in completing transfer can be ultimately reduced.

            The strategy chosen for this implementation was using a copy-through send buffer requiring every segment to be recomputed before retransmission. It has the immediate advantage of being much simpler to implement as no management of stored frames needs to occur as well as the other advantages of every sent segment being as up-to-date as possible and the lower storage overhead.

        \subsubsection{Receive Buffers}
            It is completely valid and legal to immediately discard an incoming out-of-order segment, and is the default behaviour of many primitive embedded TCP stacks. In doing so the receiving TCP is committing to the requirement that the sender \textit{must} retransmit the dropped segment as it cannot be acknowledged.

            Most competent TCP implementations queue out-of-order segments for later processing as their sequence number becomes the next awaited unacknowledged value. There are many considerations that go along with this approach such as much of the control information attached to the queued packets (e\.g\. the send window) should be considered outdated and therefore invalid. \\
            As well as data segments, control-only packets such as FIN segments should also be retained so that they do not have to be retransmitted- an out-of-order FIN should be treated the same as an out-of-order data segment.

            Much like the send buffer, the reasoning for the choice of received packet buffering was for ease of implementation, without wastefully dropping segments. All incoming segments are pushed, in sequence, into a receive queue, ready to be retrieved by a successive \texttt{recv} call by the user process. \\
            Whilst this does not allow for queuing of common out-of-order FIN control segments, it does accommodate for regular data segment reordering significantly reducing retransmission occurrences, especially in lossy environments.

    \subsection{\mbox{No specific API requirement}}
        % designed in a way that it can sit below any kind of API..
    \subsection{Language choice}
        % c is good
        % rust is probably ok
        % go is bad - Hayo 2k18

    % standard format for returning errors
    \subsection{Error Handling}
        As with most things in C, error handling must be implemented by the programmer. There are many ``standard'' methods of handling errors and they vary wildly between projects. The standard used by POSIX defined libc functions requires the use of a thread-local/global variable to track error values, known as \texttt{errno}, is roughly defined as follows:
        \begin{center}
            \begin{minipage}{0.88\columnwidth}
                \begin{itemize}[noitemsep]
                    \item{\texttt{return\ \ -1} on error, setting errno}
                        \begin{itemize}
                            \item[]{\small e.g.\ \texttt{errno = EINVAL;\ return -1}}
                        \end{itemize}
                    \item{\texttt{return = 0} on success}
                    \item{\texttt{return > 0} with value}
                \end{itemize}
            \end{minipage}
        \end{center}
        \vspace{\parskip}

        For this project I opted to use a similar but cleaner approach:
        \begin{center}
            \begin{minipage}{0.88\columnwidth}
                \begin{itemize}[noitemsep]
                    \item{\texttt{return < 0} on error, negative value}
                        \begin{itemize}
                            \item[]{\small e.g.\ \texttt{return -EINVAL}}
                        \end{itemize}
                    \item{\texttt{return = 0} on success}
                    \item{\texttt{return > 0} with value}
                        \begin{itemize}
                            \item[]{\small e.g.\ \texttt{return 20} \textcolor{gray}{{\tiny\#} number of bytes read}}
                        \end{itemize}
                \end{itemize}
            \end{minipage}
        \end{center}
        \vspace{\parskip}

        Having a well-defined standard for how errors should be handled removed any ambiguity in the code about how errors should be handled. Even though the semantics differs from that used by standard library functions, it integrated very easily with the use of a standard pattern, mapping from one to the other.

        This simpler design has two immediate advantages over the libc standard approach:
        \begin{itemize}[noitemsep]
            \item{There is no requirement for a global error storage; errors are just returned from the function call with no (extra) side-effects.}
            \item{Memory utilisation of return values is better as there is less wasted space of values less than 1. POSIX defines \texttt{ssize\_t} `to represent sizes of blocks that can be read or written in a single operation.' It must be a signed type to be able to accommodate -1, 0 and positive block sizes. On most (probably all) systems, it is defined as a signed long integer meaning any values representable less than -1 are unused.}
        \end{itemize}


%! TEX root = ../report.tex

% A detailed account of the implementation and testing of your software
% Explain the conceptual structure of the algorithms.
% Also explain what data structures you used
    % how the algorithms were implemented
% What implementation decisions did you take, and why?
    % There is no need to list every little function and procedure and explain its working in elaborate detail; use your judgement on what is appropriate to include

\chapter{Implementation}
\todo{Describe the basic packet-flow operation}

    \section{Standalone Library}

        % netstack_init()
        % netstack_cleanup()
        %

    \section{Header Manipulation}

    \section{Interfaces}\label{sec:intf}

    \section{Internet Protocol}

        \subsection{Routing}

    \section{\mbox{Address Resolution Protocol}}

    \section{Sockets}

    \section{TCP}

        \subsection{Segment Arrives}






        \subsection{Concurrency}

    \section{Socket API}
        \subsection{Netstack API}
        % TCP user functions join together the internal tcp calls and perform error checking.
        % Perform sufficient locking to prevent concurrent modification of the state whilst it's being read

            \subsubsection{Send}
            % Calls tcp_send_data
            % Waits for space in send window

            \subsubsection{Recieve}
            % Copies the largest contiguous available byte range
            % Waits for incoming segments


        \subsection{BSD Socket API}
            % socket, bind, listen, connect, send, recv, shutdown all work
            % getsockopt, setsockopt, fcntl implemented but do nothing
            % O_NONBLOCK isn't working

    \section{Memory Management}

        % refcounting
        \subsection{Reference Counting}

    % logging with macros for verbose logs
    \section{Logging}

    \section{Issues}
        \subsection{Standards}
        % RFC 793 is incomplete and very vague

        \subsection{Threading \& Locking}
        % refcount
        % pthread_cond
        % arp request waiting on recv thread

        % BSD socket API design
        % non-blocking I/O (poll/select) :(


%! TEX root = ../report.tex

% Explain your testing strategy and provide convincing examples of tests
    % In project involving software development it is usual to relate the tests to your Software Requirements Specification.

\chapter{Testing}
    % CI building
    % multiple libc impl, multiple compilers
    \section{Minimum Working Example}
        % testing against real working code for basic TCP operation
    \section{Performance Testing}
        % Profiling with callgrind (ref callgrind)
        % throughput tests
    \section{Logging}
        % useful for debugging
        % Not useful for debugging throughput


%! TEX root = ../report.tex

\section{Project Management}
    % Continuous integration
    % Kanban board with prioritised tasks
    % Too much work, not enough time for one person
        % prioritising only the most important tasks
        % DHCP, UDP, addressing, cmdline utils, configuration files all less important than TCP


%! TEX root = ../report.tex

\chapter{Results}

    % retransmissions
    % performance
        % refer to testing/minimal-ex
            % talk about transfer speed
        % with logging


%! TEX root = ../report.tex

\section{Evaluation} % Discussion
    % how it went
    \subsection{Issues that I thought were going to be but weren't actually issues}
        % refer to journal section 5
    % how it could be improved
    \subsection{Improvements}
        % send/recv buffer implementation (recv buffer structure is a better design and should be used for send buffers too)
        \subsubsection{Send/Receive Buffers}
            
        % parameterised ip/route specification
        \subsubsection{Addressing Configuration}
            % IP address, routes
            % MAC addresses
        % queue'd out-of-order packets to be re-processed for control (FIN), not just for payload
        \subsubsection{Out-of-order TCP queuing}
    % where the project should go from here
    \subsection{Future Work}
        % everything in improvements, obvs
        % TCP extensions for improved performance/throughput & reduced overhead
        % configuration
            % configurable logging
                % disable anything below particular errors
        % system-wide network daemon with IPC library
            % see journal section 4.4
        \subsubsection{Use cases}


%! TEX root = ../report.tex

\chapter{Conclusion}



\printbibliography % chktex 1

\end{document}
