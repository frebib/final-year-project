\documentclass[11pt,a4paper,british,twocolumn]{bhamreport}
\usepackage[utf8]{inputenc}
\usepackage{float}
\usepackage{listings}
\usepackage{multicol}
\usepackage{dirtree}
\usepackage{babel}
\usepackage{csquotes}
\usepackage{amsmath}
\usepackage{isodate}
\usepackage[toc]{appendix}
\usepackage[usenames, dvipsnames]{color,xcolor}

% Configure bibliography
\usepackage[backend=biber,sorting=none,citestyle=numeric-comp]{biblatex}
\usepackage[hidelinks]{hyperref}
\addbibresource{references.bib}

% Configure glossary
\usepackage[toc,acronyms]{glossaries}
\makeglossaries{}
%! TEX root = ../report.tex

\usepackage{xparse}
\DeclareDocumentCommand{\newdualentry}{ O{} O{} m m m } {
    \longnewglossaryentry{gls-#3}{name={\MakeLowercase{#4}},text={#3},#1}{#5}
    \makeglossaries{}
    \newglossaryentry{#3}{type=\acronymtype, name={#3}, description={#4}, first={#4 (#3)\glsadd{gls-#3}}, see=[Glossary:]{gls-#3}}
}

\newdualentry{API}{Application Programming Interface}{a standard set of functions/routines that perform documented actions on a specified system}
\newdualentry{BSD}{Berkeley Software Distribution}{a unix operating system derivative developed at University of California, Berkeley in the late 1970s}
\newdualentry{TCP}{Transmission Control Protocol}{a network stream protocol providing reliable, ordered, and basic integrity guarantees, part of the Internet Suite}
\newdualentry{MTU}{Maximum Transmission Unit}{the largest amount of data that can be sent in one ethernet frame}

\longnewglossaryentry{simplex}{name=simplex}{uni-directional communication}
\longnewglossaryentry{segment}{name=segment}{one packet in a TCP packet stream often containing data}

% Add all glossary items regardless of usage
\glsaddall{}



\usepackage{report}

% chktex-file 36

% Tikz for diagrams
\usepackage{tikz}
\usetikzlibrary{calc,shapes,arrows.meta,positioning}
\tikzstyle{box}   = [rectangle, minimum height=1.5em, minimum width=4em, draw=black!50, thick]
\usepackage{bytefield}

\title{A TCP/IP Networking\\Stack Implementation\\in Userland}
\author{Joe Groocock -- 1467414} % chktex 8
\supervisor{Dr Ian Batten}
\department{Computer Science}
\degree{MSci in Computer Science}

\hypersetup{pdfinfo={%
    Title={A TCP/IP Networking Stack Implementation in Userland},
    Author={Joe Groocock - 1467414}, % chktex 8
    Subject={MSci in Computer Science, University of Birmingham},
    Keywords={TCP, IP, TCP/IP, Networking, Userland}
}}

\begin{document}

\maketitle

\begingroup
\let\onecolumn\twocolumn
\tableofcontents
\endgroup

% TODO: Make abstract relevant to final project
% - it states the problem that you set out to solve
% - it describes your solution and method
% - it states a conclusion about the success of the solution
\begin{abstract}
TCP/IP network stacks included in most modern operating systems are opaque to the user and mostly unconfigurable. This paper outlines designing and implementing a performant and flexible TCP/IP standalone implementation for Unix systems. Useful as a tool for further learning, research and experimentation with various network protocols, it provides documentation and source code for a high-performance data transfer outside the kernel. Provided as a learning and academic research tool that can be tweaked, extended and analysed in depth to aid in gaining a better understanding of how operating systems handle networking and TCP/IP\@. Many common issues surrounding userspace networking are evaluated and resolved to produce a working software library and accompanying set of tools.
\end{abstract}

\section*{Acknowledgements}
Many thanks to all the support and advice from my girlfriend and housemates. Thanks to Dr Batten for his insightful advice on Unix and networking.

\twocolumn

\pagenumbering{arabic}

%! TEX root=../report.tex

\chapter{Introduction}\label{chp:intro}
% Describe the problem to be solved
The primary objective of this project is to construct a portable TCP/IP implementation running in userland with no reliance on kernel networking. Network operations in most modern operating systems are passed down to the \textit{kernel} for processing, to maximise performance.

    % Explain enough background to understand the problem
    % Clearly state project aims
    \section{Aims}\label{sec:aims}
    Learning about the inner workings of TCP, IP and other protocols with the black-box approach using the kernel implementations is difficult. Producing modifications to interwork with existing code requires compilation of some or all of the kernel in order to be applied, impeding productivity. Many of these drawbacks can be overcome by moving the networking code up into userspace. Requiring only a single passage to transfer complete ethernet packets back and forth to the physical network interface, a network software utility can be entirely self-sufficient and thus much more manageable.

    Providing a software library to perform network protocol operations opens up many opportunities for further research into userland adaptations of existing protocols or prototyping new protocols with greater ease. Access to the inner workings of protocols has potential to allow for new types of network packet inspection or injection, for example a \textit{virtual private network} tunnelling implementation that relies solely on packet-level access for data injection.

    As a learning tool it is important that it performs correctly, as per the specification, and is performant enough to be useful. Many implementations are littered with additional functionality for legacy systems or rarely-used features that lessen the overall learning experience. Code should be well commented and readable whilst still not sacrificing performance.

    \section{Structure}
    % Provide an overview of structure of solution?
    Making up the whole solution to this project are multiple modular components that when combined can be used in many different ways, for example some of the provided use cases in Section~\ref{sec:aims}.
    Most importantly is the network stack implementation which provides an end-to-end process for handling TCP, IP and other protocols. The network stack is comprised of many layers, as discussed in-depth in Chapter~\ref{chp:design}. To integrate seamlessly with existing software, \texttt{libnshook} and \texttt{libnetstack} can be used to replace the default networking in Unix application binaries.

    % Explain what is in each section of this document
    In Chapter~\ref{chp:spec} the project goals are formalised into requirements. Chapter~\ref{chp:background} overviews the history and operation of TCP and IP protocols. Discussion of other works, relating to the topic of this project are covered in Chapter~\ref{chp:related}. Detailed information about how the project was designed and implemented are covered in Chapters~\ref{chp:design}~\&~\ref{chp:impl}. An overview of the testing tools and strategies used is in Chapter~\ref{chp:testing}. The project outcomes and experiences are summarised and evaluated in Chapter~\ref{chp:eval}.


%! TEX root = ../report.tex

\chapter{Specification}\label{chp:spec}
    % How you analysed the problem
    % Give an appropriate specification of the solution

    % Define specific behavior or functions
    % what a software system should do
    \section{Functional Requirements}
    \begin{itemize}
        \item This project must produce a TCP/IP implementation that runs completely in Linux userspace with no kernel modification.
        \item This project must follow the original RFC 793~\cite{rfc793} specification for TCP.
        \item This project could implement newer TCP extensions or improvements.
        \item The implementation should be portable across Linux and Unix-like systems.
        \item This project must provide a network stack as a library such that it can be easily implemented in other software.
        \item This project should provide functionality to inject the network stack into existing applications.
        \item Customisability and plugability could be implemented to allow expansion and/or modification of new and existing protocols.
        \item Adapting the library or binaries should be flexible and should provide various modes of operation to suit a wide range of systems.
    \end{itemize}

    % A requirement that specifies criteria that can be used to judge the operation of a system, rather than specific behaviors.
    % constraints on how the system should do so
    % elaborates a performance characteristic
    \section{Non-functional Requirements}
    \begin{itemize}
        \item The code should be sufficiently commented such that it can be easily understood, modified and learned from.
        \item Performance of data transfers within the TCP subsystem should be able to sustain at least 100 megabit/second average throughput to stay relevant in modern network scenarios.
    \end{itemize}

    % What the user expects the software to be able to do
    \section{User Requirements}
    To be able to quantify requirements for a user, first the different types of users of the application must be defined.

        \subsection{Student}
        Using the project as a learning tool demands in-depth documentation and clear, commented code. Users will need to be able to follow through the code and cross-reference it with the logs as it runs. Running the software against existing programs and linking with new programs should be as simple as using any other standard dynamic C library.

        \subsection{Researcher}
        Customisability of internal protocols should be straight-forward with minimal code refactoring required to perform the desired changes. The possibility of adding new or modified protocols into the existing stack should require only the addition of only a few lines of code without having to refactor the layering to accommodate new protocol designs.


%! TEX root=../report.tex

\chapter{Background}\label{chp:background}
%! TEX root = ../report.tex

\section{TCP/IP Model}
There are many varying interpretations of the TCP/IP network model and the layers it consists of. According to RFC 1122~\cite{rfc1122} there are 4 layers, presented as shown in Figure~\ref{fig:tcpip-model}.

\begin{figure}[H]
    \begin{center}
        \setlength{\tabcolsep}{2pt}
        \begin{tikzpicture}[
            node distance=0em,
            every node/.style={%
                inner sep=0.7em,
                outer sep=0,
                draw=black!80,
                thick
            }
        ]
            \fontsize{9pt}{9pt}
            \node [minimum width = 10em, minimum height = 3em, fill=yellow!50]                    (applicat)  {Application (4)};
                \node [right = of applicat, xshift=1em, fill=yellow!30] (http)  {HTTP};
                \node [right = of http, fill=yellow!30]     (ssh)   {SSH};
            \node [minimum width = 10em, minimum height = 3em, fill=ForestGreen!50, below = of applicat] (transport) {Transport (3)};
                \node [right = of transport, xshift=1em, fill=ForestGreen!30] (tcp)  {TCP};
                \node [right = of tcp, fill=ForestGreen!30]     (udp)   {UDP};
            \node [minimum width = 10em, minimum height = 3em, fill=cyan!50, below = of transport](internet)  {Internet (2)};
                \node [right = of internet, xshift=1em, fill=cyan!30] (ipv4)  {IPv4};
                \node [right = of ipv4, fill=cyan!30]     (ipv6)   {IPv6};
            \node [minimum width = 10em, minimum height = 3em, fill=purple!50, below = of internet] (network)   {Network/Link (1)};
                \node [right = of network, xshift=1em, fill=purple!30] (eth)  {Ethernet};
                
            \node [draw=none,below = of network,yshift=-1em] (layers)   {\textit{Layers}};
            \node [draw=none,right = of layers,xshift=4em]              {\textit{Protocols}};
            \draw [->,gray!80,thick] ([xshift=-1.2em]network.west)  -- ([xshift=-1.2em]applicat.west) 
                    node[midway,draw=none,text=gray,above,rotate=90] {Abstraction};
        \end{tikzpicture}
    \end{center}
    \caption{TCP/IP network model}\label{fig:tcpip-model}
\end{figure}

Each layer provides one or more protocols that define guarantees about the data they carry. The higher layers provide protocols that are more capable and complex than the layer below. Each layer also accumulates the features of the layer below it. For example the \textit{internet} layer provides host addressing which can be leveraged from the transport layer.

This project implements the lowest three layers of the TCP/IP model and provides an API to interface with them from the application layer.



%! TEX root = ../report.tex

\section[IP Overview]{Internet Protocol Overview}\label{sec:intro-ip}
The Internet Protocol (IP) provides global addressing across networks. IP mandates that every interconnected host must have an IP address to be able to communicate. It also ensures that any interconnected hosts can communicate with any other interconnected host, given their address through the process of routing, regardless of the physical distance or presence of barriers. The largest IP network is the `internet' where any device with an address can (in theory) inter-communicate with any other device.

% Talk about IPv4 vs IPv6
To date there have been two prominent versions of IP, both of which are still in widespread used today across the internet and private intranets alike. IP Version 4, first published in \citeyear{cerf1974protocol} by \citeauthor{cerf1974protocol}~\cite{cerf1974protocol} was the first non-experimental datagram protocol in the Internet Suite. It is still the most widely used protocol today almost 45 years on.

Following the massive success of IPv4 is version 6 of the IP suite. Drafted as a standard in 1998 to replace IPv4 and to resolve the impending IPv4 address exhaustion problem, boasting significantly more usable host addresses and many smaller incremental improvements over the previous version 4~\cite{hovav2004model}. Only reaching maturity as an official `Internet Standard' in mid 2017, acceptance and adoption has been slow but is picking up much needed traction to eventually match and finally overtake the dated and problematic IPv4 in coming years.

    % Compare addressing schemes
    \subsection{Addressing}
    IPv4 packets require 32 bits (4 bytes) for host addresses, one for each of source and destination. Given every combination of those values, it results in roughly 4.3 million address ($4.3 \times 10^{9}$ \textit{OR} $2^{32}$). Due to reserved groups and lack of foresight during the design phase, only about 3.7 million of those addresses are usable due to reserved allocation such as those defined in RFC 1918~\cite{rfc1918} and other wastage of various kinds.

    IPv4 implements `Network Address Translation' (NAT) which is a hack to compensate for the ever-growing demand for addresses, even after there are no more available. NAT works by translating one or more public facing addresses to many private addresses, bidirectionally. It has been modified and improved as it has evolved to allow many clients make simultaneous outgoing connections from behind the same IP address and even accept incoming connections given the correct configuration.

    IPv6 addresses one of the major caveat in version 4 of the protocol with much larger address values of 128 bits which results in roughly 340 undecillion ($3.4 \times 10^{38}$ \textit{OR} $2^{128}$) usable addresses. The available addressable space is so vast that currently only a small portion is available for public use and the smallest division that can be allocated by a provider is $2^{64}$ addresses.

    \subsection{Subnets}
    Subnets are divisions of the whole IP space, useful for allocating sections to the entities that own rights to them. Traditionally, subnets were defined `classfully' where the value of the address defined how large the network it belonged to was. Primarily there were three classes of varying sizes:
    \begin{itemize}
        \item{Class A occupied half of the IPv4 space ranging from \texttt{0.0.0.0/8} to \texttt{128.0.0.0/8}, consisting of 256 networks of $2^{24}$ hosts each.}
        \item{Class B defined $16,384$ networks, ranging from \texttt{128.0.0.0/16} to \texttt{191.0.0.0/16}, each with $65,536\ (2^{16})$ hosts.}
        \item{Class C boasted $2,097,152\ (2^{21})$ networks, each with only $256\ (2^{8})$ hosts.}
    \end{itemize}

    It was originally planned to delegate various network segments to companies based on their requirements. Large enterprises such as DARPA, Ford, HP and Apple (at the time) were allocated a class A network each, whilst smaller companies were allowed class B and C portions.

    Introduced in 1993, `Classless Inter-Domain Routing' (CIDR) was introduced to permit more flexible network separation with the use of a \textit{variable-length subnet mask}. CIDR is denoted it two forms, either a binary representation such as \texttt{255.255.255.0} or as a decimal value preceded with a slash \texttt{xxx.xxx.xxx.xxx/24}. The value represented by the CIDR refers to the amount of bits that define the `network' portion of the address. The remainder of the bits in the address refer to the hosts in that network.

    CIDR has for the most part replaced classful networks and is used in both IPv4 and IPv6 for defining network subdivisions.

        \subsubsection{Subnetting example}
        Given the typical example host address \texttt{192.168.8.99/24}, similar to one that may be found in consumer hardware used in a small home network, the following information can be extrapolated:
        \begin{itemize}[noitemsep]
            \item{The host address is \texttt{192.168.8.99}}
            \item{The `network address' is \texttt{192.168.8.1}}
            \item{The `broadcast address' is \texttt{192.168.8.255}}
            \item{The `subnet mask' is \texttt{255.255.255.0}}
            \item{There are 254 \textit{usable} host addresses}
            \item{Hosts between the addresses \texttt{192.168.8.1} and \texttt{192.168.8.254}, inclusive, can be communicated with directly, without the need for a router/gateway}
        \end{itemize}

    % Explain what it is an how it works. interconnected graph, with routers
    \subsection{Routing}
    Communications are exchanged between internet hosts through a process called `routing'. Packets are dispatched from the source and propagate through one router at a time until they arrive at their destination.
    
    Routes are defined per-interface with a combination of an IP address and a `subnet mask' in CIDR form. Packets can be either delivered directly to the destination host if it resides within one of the local subnets, or via a router.
    In the case of local addresses, the hardware address of the destination host is used for the \textit{link-layer} protocol, which is usually ethernet, and the packet is sent directly to that host.
    Alternatively, if there is no local route available, the hardware address of the specified gateway host is used for the \textit{link-layer} instead and the packet is sent to the gateway, in the hope that it will forward the packet on to it's destination in another network.

    Routers are internet hosts whose primary responsibility is to forward packets from one network to another. Often routers will know which networks they are neighbouring so can with reasonable accuracy choose a short route for any given packet. In the case of large networks such as the internet, packets will `hop' from one router to the next, following the routes that match the destination IP address until the packets reaches the destination subnet and ultimately the destination host addressed by the packet.

    IP makes no guarantees about packet delivery, instead it operates on a `best effort' policy; all a router can do is attempt to forward the packet on assuming that it will eventually arrive at it's final destination.

    End-user clients such as personal computers attached to small private networks typically are only aware of a single `route', via their local \textit{gateway} which are dynamically assigned when the device connects to the network. A gateway is a router on the local subnet that can forward a packet toward it's final destination by one \textit{hop}.
    Conversely, routers are largely interconnected devices which are permanently connected to many networks simultaneously. Known routes will be statically assigned, either manually or autonomously by a routing protocol, and forward packets between the various subnets, in effect connecting them together.

%! TEX root = ../report.tex

\section[TCP Overview]{Transmission Control Protocol Overview}
    This overview mostly covers the essentials as defined by the original specification, RFC~793~\cite{rfc793}. Extensions or variations are very useful for improving performance and reliability but absolutely not required for a functioning \gls{TCP} implementation. % chktex 13

    \subsection{Terminology}
        \begin{itemize}
            \item \texttt{SYN}: Synchronise flag\\
                \textit{Set to establish a new connection}
            \item \texttt{FIN}: Finish flag\\
                \textit{Indicate `no more data to send', closing existing connection}
            \item \texttt{ACK}: Acknowledgement flag\\
                \textit{Confirming that data arrived successfully or acknowledging a command}
            \item \texttt{RST}: Reset flag\\
                \textit{Sent in reply to immediately terminate any connection}
            \item Bit field: \textit{Binary data where each individual bit represents a boolean value e.g. 0 or 1}
        \end{itemize}

    \subsection{TCP packet header}
        A \gls{TCP} packet, or segment, contains an initial header followed by the packet payload. The header can be a variable length but is always at least 20 bytes and at most 60 bytes so small payloads are an inefficient use of a segment as there is always a 20+ byte overhead per packet. The header is always a multiple of 32 bits (4 bytes) which is the \textit{word size} for the \gls{TCP} header, and is padded with zeros to fill any remaining space.

        Most of the information carried in the header is metadata for identifying the connection and the data within the packet. Many of the values such as \textit{sequence number} and \textit{window size} correspond to the sender; each peer will have their own values.

        \bigskip
        \begin{figure}[H]
            \centering
            \fontsize{8pt}{10pt}
            \begin{bytefield}[bitheight=2em,bitwidth=0.028\columnwidth]{32}
                \bitheader{0,4,8,10,12,14,16,24,31} \\ % chktex 8
                \bitbox{16}{Source Port} & \bitbox{16}{Destination Port} \\
                \bitbox{32}{Sequence Number} \\
                \bitbox{32}{Acknowledgement Number} \\
                \bitbox{4}{Data offset} & \bitbox{6}{Reserved} &
                \bitbox{1}{\tiny \rotatebox{90}{URG}} & \bitbox{1}{\tiny \rotatebox{90}{ACK}} &
                \bitbox{1}{\tiny \rotatebox{90}{PSH}} & \bitbox{1}{\tiny \rotatebox{90}{RST}} &
                \bitbox{1}{\tiny \rotatebox{90}{SYN}} & \bitbox{1}{\tiny \rotatebox{90}{FIN}} &
                \bitbox{16}{Window} \\
                \bitbox{16}{Checksum} & \bitbox{16}{Urgent Pointer} \\
                \bitbox{24}{Options} & \bitbox{8}{Padding} \\
                \wordbox[lrt]{1}{Payload~\dots} \\
                \skippedwords[0.8em]{} \\
                \wordbox[t]{1/2}{}
            \end{bytefield}
            \vspace*{-1em}
            \caption{\gls{TCP} Packet header}\label{fig:tcphdr}
        \end{figure}

        %In order of appearance, the fields in the header are as follows:
        % \todo{move tcp bits to glossary or appendix or something}
        % \begin{itemize}
        %     \item \textbf{Source Port} \textit{2 bytes} Port from which this packets was dispatched.
        %     \item \textbf{Destination Port} \textit{2 bytes} Port number open to receive on the recipient device.
        %     \item \textbf{Sequence Number} \textit{4 bytes} Sequential count of the first byte in the payload, identifying the sequence position of the segment.
        %     \item \textbf{Acknowledgement number} \textit{4 bytes} Sequential count of successfully received bytes.
        %     \item \textbf{Data offset} \textit{4 bits} Size of the \gls{TCP} header in 32-bit words, as the header is padded to multiples of 4 bytes. This indicates the start of the payload and the end of the header within the packet.
        %     \item \textbf{Flags} \textit{12 bits} A \textit{bit field} containing the connection control flags.
        %     \item \textbf{Window} \textit{2 bytes} Size, in bytes, of the receive window available to fill.
        %     \item \textbf{Checksum} \textit{2 bytes} The \textit{internet checksum}~\cite{internetchecksum} of the entire \gls{TCP} segment and the upper \textit{Internet Protocol} layer. Used to ensure (probable) integrity of the data from corruption.
        %     \item \textbf{Urgent Pointer} \textit{2 bytes} Indicates the offset from the \textit{Sequence number} of the last urgent byte when the \texttt{URG} flag is set.
        %     \item \textbf{Options} \textit{0-40 bytes} Optionally, a series of \gls{TCP} extension options followed by padding to the nearest 4th byte.% chktex 8
        %     \item \textbf{Payload} \textit{0-MSS bytes} Sequenced data for the \gls{TCP} connection. Can be empty. Limited by the \textit{MSS} of the connection.
        % \end{itemize}

    \subsection{Connection-oriented}
        One of the defining features of \gls{TCP} is the \textit{three-way handshake}~(Figure~\ref{fig:threeway}) which every connection begins with.

        \begin{figure}[H]
            \centering
            \begin{tikzpicture}[>=latex,scale=1.2]
                \footnotesize
                \coordinate (TL) at (1.2,3);
                \coordinate (BL) at (1.2,0);
                \coordinate (TR) at (3.4,3);
                \coordinate (BR) at (3.4,0);
                \coordinate (LA) at ($(TL)!.25!(BL)$);
                \coordinate (LB) at ($(TL)!.65!(BL)$);
                \coordinate (LC) at ($(TL)!.70!(BL)$);
                \coordinate (LD) at ($(TL)!.75!(BL)$);
                \coordinate (RA) at ($(TR)!.14!(BR)$);
                \coordinate (RB) at ($(TR)!.40!(BR)$);
                \coordinate (RC) at ($(TR)!.45!(BR)$);
                \coordinate (RD) at ($(TR)!.50!(BR)$);
                \coordinate (RE) at ($(TR)!.90!(BR)$);
                % Lines & labels
                \draw (TL) node[above]{\large Client};
                \draw (TR) node[above]{\large Server};
                \draw[thick] (TL)--(BL) (TR)--(BR);
                % LISTEN
                \draw (RA) node[right]{%
                    \begin{tabular}{l}
                        \verb$LISTEN$\\
                        \textit{passive open}
                    \end{tabular}
                };
                % \texttt{SYN}
                \draw (LA) node[left]{%
                    \begin{tabular}{r}
                        \textit{active open}\\
                        \verb$SYN_SENT$
                    \end{tabular}
                };
                \draw[->] (LA) -- (RB) node[midway,sloped,above]{\verb$SYN$};
                \draw (RC) node[right]{%
                    \begin{tabular}{l}
                        \verb$SYN_RCVD$\
                    \end{tabular}
                };
                % \texttt{SYN}/\texttt{ACK}
                \draw[->] (RD) -- (LB) node[midway,sloped,above]{\verb$SYN/ACK$};
                \draw (LC) node[left]{%
                    \begin{tabular}{r}
                        \verb$ESTABLISHED$\
                    \end{tabular}
                };
                % \texttt{ACK}
                \draw[->] (LD) -- (RE) node[midway,sloped,above]{\verb$ACK$};
                \draw (RE) node[right]{%
                    \begin{tabular}{l}
                        \verb$ESTABLISHED$\
                    \end{tabular}
                };
            \end{tikzpicture}
            \caption{\gls{TCP} Three-way handshake}\label{fig:threeway}
        \end{figure}

        A client requests a new connection by sending an initial packet with the \texttt{SYN} bit set, as well as an initial random \textit{sequence number}. When a packet marked with \texttt{SYN} is received at the server, a new connection is being established and the server replies setting both \texttt{SYN} and \texttt{ACK} bits in the packet. This reply does one of two things:
        \begin{enumerate}
            \item The mutual \texttt{SYN} flag indicates the connection is being accepted along with a separate random sequence number.
            \item The \texttt{ACK} flag being set along with the \textit{acknowledgement number} indicates it understood the initial \textit{sequence number}.
        \end{enumerate}
        The sequence number from the initial flag is reused, incremented by one and sent back as the acknowledgement number in the \texttt{SYN}/\texttt{ACK} packet.

        Finally, the client replies with the sequence number sent by the server, incremented by one, which completes the three-way handshake. At this point the connection is set up and ready to be used to transmit data.

        When the connection is finished with, both endpoints need to mutually close the connection, in a very similar fashion to the \texttt{SYN}, \texttt{SYN}/\texttt{ACK}, \texttt{ACK} pattern.
        Either side of the connection can initiate a \texttt{FIN} at any point, indicating there is no more data to send to which the peer must respond with an \texttt{ACK} of the final sequence number incremented by one to confirm the \texttt{FIN} packet was received.

        \texttt{FIN} packets can behave slightly differently to \texttt{SYN} packets as there is no requirement to send a \texttt{FIN} packet immediately after receiving one. In fact, a connection can remain \textit{half-closed} for some time whilst one side finishes transmitting data. This is a common pattern for some software applications as only \gls{simplex} communication is required. It is also acceptable to close a connection quickly with \texttt{FIN}, \texttt{FIN}/\texttt{ACK}, \texttt{ACK}.

        If at any point a \texttt{SYN}, \texttt{FIN} or other packet has not been acknowledged by the peer within a certain time frame, the packet will be resent under the assumption it never arrived; this is part of the reliability of \gls{TCP} (See Section~\ref{sec:reliable}).

        Following this pattern ensures both peers are in agreement of the state of the connection at all times. If at any point an invalid connection request is sent, to a closed port for example, it is instantly replied to with an \texttt{RST} packet indicating that the client should immediately discard the connection as it is invalid.

    \subsection{Ordered}
        \textit{Sequence numbers} are used to track each individual byte sent over the connection. For each byte successfully received by the peer, the acknowledgement number is incremented to that value. Each packet contains a sequence number for the first byte of the payload in that segment indicating the position of the data in the continuous stream. If a packet arrives out of order, it can be kept aside until it is required or dropped and it will be retransmitted. Data that has been sent but not acknowledged is assumed to be lost so, after a timeout, it is retransmitted.

        For the most part, especially with small local networks or across short distances, packets will arrive in order and complete so little retransmission is required, just some reordering and payload reassembly.

        There are various methods of tracking missing packets, storing and reordering packets, indicating that packets are lost to reduce retransmission latency and so on that are used in \gls{TCP} implementations which can improve performance and throughput in many cases, however, they are not required for basic \gls{TCP} functionality.

    \subsection{Reliable}\label{sec:reliable}
        Acknowledging data is the first method used by \gls{TCP} to ensure that all data arrives intact. The sender tracks the sequence number of the last acknowledgement it received. It also buffers the transmitted but unacknowledged data in the case of it having to be retransmitted. As data is received at the client, acknowledgements are sent in reply confirming the data was received.

        Complementing that, a primitive checksum known as the \textit{Internet Checksum} is used to verify integrity of the data. The \textit{Internet Checksum} is defined as ``the 16 bit one's complement of the one's complement sum of all 16 bit words in the header and text''~\cite[3.1]{rfc793}.

        By modern standards it is a very poor checksum design, however it is very fast to compute and is still sufficient to ensure that no serious data corruption has occurred. In most cases, even if the checksum is still valid, corruption caused to the header could be detected and the packet will be discarded likely due to out-of-band sequence numbers or similar. As the checksum is simple in nature, it can only provide basic data integrity and does not protect against malicious attacks in the way that cryptographic hashes do.
        
        \subsection{Extensible}
        \gls{TCP} has a vast array of extensions, specified in the `Option' section at the end of the packet header, that are used to fix many issues, improve performance and add extra functionality. This section covers the commonly used options of note, but there are many more that are widely used and some still in development.

        \subsubsection{Window Scaling}\label{sec:wscale}
            At the advent of \gls{TCP}, networks were comparatively slow at only a few Megabits per second, however, technology has advanced and now links are available in excess of 100 Gigabits per second. The default \textit{Window size} defined in \gls{TCP} is only 64KiB which at 100Gbps can be filled in just 5 microseconds. Small receive windows can slow down transmission dramatically as the buffer has to be cleared before it can be populated again by the sender. To leverage these high transfer speeds, Window Scaling~\cite[2]{rfc1323} was introduced to increase the size of the default receive window.

            Window scaling must be negotiated during the initial handshake of the connection; a client requests \texttt{wscale} in the first \texttt{SYN} packet to which the peer replies with a \texttt{wscale} option in the returning \texttt{SYN}/\texttt{ACK} segment, confirming the choice to use scaling for the remainder of the connection. Should no reply to the \texttt{wscale} option be sent, it is assumed the peer does not support window scaling and therefore the connection must operate without scaling.

            The option specifies a scale value between 0 and 14 which should be applied as a \textit{bitwise left shift} to the window size allowing a maximum receive window size of 1GiB ($65536 << 14$). It is completely valid to negotiate a `zero' shift in one direction to disable scaling for one client.

        \subsubsection{Selective Acknowledgement}\label{sec:sack}
            Acknowledgements are part of the reason \gls{TCP} is so useful, but in particularly poor network conditions the performance of \gls{TCP} can suffer from excessive retransmissions, driving latency up and throughput down. Often in lossy transfers only a small number of packets will be lost but \gls{TCP} defines that data should be acknowledged until that point and all data afterwards will be retransmitted. It is often the case that most or all data after a missing packet will have arrived successfully but would be unnecessarily retransmitted.

            Selective \texttt{ACK} aims to reduce and abolish retransmissions for data that has already arrived by indicating to the sender which blocks have or have not arrived. It is an option appended specifically to \texttt{ACK} packets to give more detail about which byte ranges should be retransmitted. The option has a variable payload length to allow inclusion of the left and right boundaries of each block that is being acknowledged.

        \subsubsection{\citeauthor{rfc896}'s Algorithm}
            \citeauthor*{rfc896} proposed in \citeyear{rfc896}~\cite{rfc896} the concept of reducing the amount of small packets being transmitted by buffering outgoing data for a small period of time before sending it. This is today known as \textit{Nagle's Algorithm} which is defined to address the ``small packet problem'' by not transmitting new data whilst there is still unacknowledged data in transmission. Sent data is queued indefinitely in the transmit buffer until an acknowledgement is received or until the buffer contains enough data to fill at least one segment.

            The segment size in this case is the \textit{Maximum Segment Size} calculated by the lowest \gls{MTU} in the route between the two communicating nodes, minus the header sizes. The MSS tends to be 1460 bytes, or less, assuming the standard Ethernet (IEEE 802.3) \gls{MTU} of 1500 is used.

        \subsubsection{TCP Corking}
            Linux implements a feature typically referred to as \texttt{\gls{TCP}\_CORK} which is similar in nature to the operation of Nagle's Algorithm. The ultimate goal of corking is the same as Nagle, but is achieved in a different way. Corking will always buffer data, even after the reception of an acknowledgement, until there is enough to fill a segment or until the timeout has passed.  Linux uses a default corking timeout of 200ms.

        %\subsubsection{Congestion Control}
            %One of the final main features of \gls{TCP} is its ability to survive and recover from network congestion with the use of complex control algorithms. Congestion control actually refers to a set of tools that a \gls{TCP} should have, discussed in detail in the following sections:

        \subsubsection{Slow Start}
            A connection between two nodes is limited by the speed of the slowest intermediary node, such as a router, network interface or switch. To avoid overwhelming any of these devices resulting in packet loss, a \gls{TCP} should assume the worst-case and start transmitting slowly. Congestion controlled \gls{TCP} implementations maintain \textit{congestion window} which is the representation to the sender of the state of congestion in the network between the connected peers.

            Slow start initialises a small congestion window, usually a small multiple of the \textit{maximum segment size}. Each time data is transmitted, the sender only releases at most the amount specified by the congestion window. For each acknowledgement received for all transmitted data, the congestion window is increased, often by a factor of two. Each subsequent transmission will be larger than the previous as the congestion window increases. When a lost segment is detected, slow start will stop and the standard linear growth Congestion Avoidance algorithm will kick in.

        \subsubsection{Congestion Avoidance}
            There have been many varying implementations of Congestion Avoidance algorithms with ranging feature sets for different scenarios. Some offer fast recovery, smaller loss or improved throughput. The aim of these algorithms is to continue a stable transmission between the two hosts whilst preventing subsequent packet loss by over-sending and overloading the network.

            Popular algorithms for this are: Vegas, BIC, CUBIC, Westwood, Reno, Tahoe and New Reno. Typically `New Reno' is used in Unix systems today as it provides the best performance and stability with minimal changes required to either sender or receiver.



\input{sections/related-work.tex}
\input{sections/design.tex}
\input{sections/implementation.tex}
%! TEX root = ../report.tex

% Explain your testing strategy and provide convincing examples of tests
    % In project involving software development it is usual to relate the tests to your Software Requirements Specification.

\chapter{Testing}\label{chp:testing}

    % CI building
    \section{Continuous Integration}
    Continuous Integration (CI) is a useful tool that when used correctly can handle the repetitive tasks such as compiling software and running simple tests. One important software development rule when publishing source code to a public repository is to ensure every commit builds and runs. Trying to maintain this high standard whilst also juggling uncommitted changes for other features can be taxing.
    % multiple libc impl, multiple compilers
    For every commit, a CI server compiled the project using both \texttt{gcc} and \texttt{clang} compilers against both GNU and musl \texttt{libc} implementations for the best interoperability on Linux systems. Testing against other Unix operating systems was not available. Provided along with each build is a log of build warnings and errors that are flagged for failing builds, so that they can be quickly fixed and amended. The simple build testing brought to light several issues relating to missing code or incomplete merges.
    
    Functionality for running and validating unit tests was available too within the CI system however only two minor test suites were written and provided little extra help once the tests passed. Not enough time was available to allow for the addition of unit tests for the whole library.

    % unit/integration tests would have been useful to also taken too long to write and manage

    \section{Testing Strategies}
    The majority of time spent programming has been towards building a working TCP implementation that will interwork with existing systems, as it is the easiest method of testing expected behaviour. Using Linux as the known-good implementation to observe and respond to, as it was readily available, made debugging strange behaviours simpler. Through the wide array of network debugging and testing tools available on Unix systems such as tcpdump and nping, crafting specific packet patterns and observing real-time reactions of both this project and reference implementations was trivial.

    % testing against real working code for basic TCP operation
    As a basic TCP connection tool, netcat was an essential peer client and server for debugging TCP data transfer problems. Paired with GNU coreutils, netcat allowed quick construction of TCP connections, sending and receiving specially crafted payloads to thoroughly and repeatably test required functionality of the TCP implementation.

    % repeated success
    Network transmission is renowned for unreliability and as a result it was important to thoroughly test code paths multiple times to ensure sound implementation and adherence of the TCP specification. Many issues, relating to concurrency and locking, only appeared very rarely from race conditions in the code. Repeated and directed tests towards these problems often highlighted issues quickly.

    % many tests will succeed most of the time but fail occasionally
    % edge-cases are VERY common and hard to spot
    Lengthy tests were performed to ensure connection endurance and continuous transfer. Tests comprised of transferring multiple gigabytes of data at once were performed repeatedly to ensure connections both performed well and could sustain heavy load and still produce identical data signatures at both ends of the connection. Many tests ended up slowing to a crawl, crashing due to memory errors or endlessly allocating memory until the computer ground to a halt. Most of the problems identified by these tests were either severe issues rendering the software useless or were rare edge-case bugs that wouldn't have been identified without them.

    \section{Minimal Working Example}\label{sec:test-working-example}
    A realistic goal was to reliably transfer at least enough data to completely wrap the TCP sequence space at least once.
    The test payload used was 4.1 GiB in size, just slightly larger than the 4.0GiB of the TCP sequence number space. It was transferred in both directions between netcat on Linux on a remote device and this project performing the same basic read/write operation on the other. Both devices connected to the same subnet, communicating through only switches and no routers.

    A file containing random data was generated on the local device and then loaded into a the network stack to be sent. \texttt{tmpfs} was used for file backing on both the sender and receiver to reduce filesystem overhead.

    \texttt{\small\textcolor{ForestGreen}{local \$} base64 /dev/urandom | head -c 4402341478 > /tmp/randfile}

    A cryptographic hash of the data was taken using \texttt{sha256sum}, to ensure integrity of the file.

    \texttt{\small\textcolor{ForestGreen}{local \$} sha256sum /tmp/randfile\\3539f4ed21b052dbcc717d1a7dcfe97fd2ca06i4\ldots}

    For the test, a simple C program using sockets was written to open, read and transfer the file, much the same way that netcat operates.\\
    The remote device was configured to listen for the incoming file and to write it to disk

    \texttt{\small\textcolor{RoyalBlue}{remote \$} nc -Nl 2345 <\&- > /tmp/randfile\\4.10GiB 0:01:52 [37.2MiB/s] [  <=>  ]}

    Finally after the file transfer had completed, a checksum of the received file was taken to ensure the file had arrived exactly as it was transmitted.

    \texttt{\small\textcolor{RoyalBlue}{remote \$} sha256sum /tmp/randfile\\3539f4ed21b052dbcc717d1a7dcfe97fd2ca06i4\ldots}

    This test methodology provides several important facts about transferring large amounts of data between Linux and this network stack implementation, in no particular order:
    \begin{itemize}[noitemsep]
        \item{Transfers totalling to amounts larger than the total TCP sequence space can address ensures that wrapping sequence numbers works as expected.}
        \item{Data is transferred in order and segmented/reassembled correctly.}
        \item{Given the correct environment, data transfer be fast, even in the worst-case scenario.}
        \item{Interoperability with Linux, and in theory other implementations, is working correctly.}
    \end{itemize}

    \section{Performance \& Bug Testing}\label{sec:perf-bug-testing}
    % memory leak checking with valgrind/memcheck
    Working with high-speed data transfer can quite quickly result in large memory leaks, even in simple programs. It is paramount that no memory is leaked at any point in the application lifecycle. Valgrind~\cite{valgrind} with \texttt{memcheck}, a memory checking tool, was incredibly helpful in identifying and resolving memory leaks. In particular it was also helpful for identifying crashes due to memory errors, often due to incorrect pointer arithmetic.

    % thread error checking with valgrind/drd
    % Profiling with valgrind/callgrind (ref callgrind)
    Valgrind also provides other tools for testing correctness of other aspects of a program. DRD, a thread error detector, was heavily utilised for identifying locking errors, especially deadlocks and regions of code without locking. Callgrind, a profiling tool, was useful to identify slow or busy sections of the program code, by function.

    Having the power to observe and inspect memory accesses at real time comes with a large performance penalty. In most cases the resulting performance was $10 \mhyphen 100 \times$ slower than standard operation. In most cases the speed was not an issue as once the bugs were identified and fixed, the program could be re-run at full speed without the memory monitoring.

    \section{Logging}\label{sec:logging}
    Profiling with Callgrind showed that up to 25\% of the computation time whilst transferring a small file was spent formatting and writing log entries. Logging is an essential tool for debugging any application as it allows the internal state of the application to be viewed by the user in real time.

    % useful for debugging
    This project makes extensive use of logs to produce informative output with data and other beneficial information about the operation of the process. As well as pre-defined messages, log entries are.colour coordinated according to their severity. Tracking bright red and purple log entries in a stream of fast flowing messages has proved to be much simpler than attempting to identify serious issues from just the text alone.

    % Not useful for debugging throughput
    When attempting to debug particular issues related to high-speed transmission, such as timing problems, the logs become more of a hindrance than a useful tool as they have a significant impact on data transmission speed. Much like Valgrind, the insight provided comes a cost of performance. Peak performance of $> 300$ megabits/s when logging is disabled becomes around 1 tenth of that when logging is enabled, depending on where the logs are written to. Nullifying the logs by writing them to \texttt{/dev/null} yields performance very close to that of no logging, only from the overhead of formatting the unused log messages and the write system calls.

    Some performance could be retained when using a fast in-memory buffer for log messages, such as `Alacritty', a GPU accelerated terminal emulator written in Rust. Typically, throughput of $100 - 120$ megabits/s could be achieved which is a $4 \mhyphen 8 \times$ improvement over writing logs to file or a standard terminal.


    \section{Tracing Packets}
    % internal logging
    Every packet, incoming or outgoing, that is processed by the network stack is logged. A breakdown of the protocol layers are produced. Initially, observing packet information was used for detecting corrupt packets or incorrectly formed segments and used as an aid to identify bugs in packet construct routines. Throughout debugging and testing the higher-level TCP protocol, the breakdown of packet information was invaluable to ensure code correctness and real network packets identify as the code should have produced them.

    % tcpdump
    Most throughput-induced issues had to be debugged blind, with no logging and no checking tools as it was the most reliable method of repeating such bugs. One tool that had little impact on networking performance whilst still giving a reasonable insight into the operation of the network stack is tcpdump. As well as providing basic packet information, including a detailed breakdown of the protocols, such as IPv4 and TCP, accurate timestamps are also given.
    Using this information it was often viable to infer what the problems were caused by with just the packet traces and intuition by cross-referencing the observed behaviour with the code paths that should have caused it.


\input{sections/evaluation.tex}
%! TEX root = ../report.tex

\onecolumn

\chapter{Conclusion}
% produced a self-sufficient network utility that runs only in userspace
% produced a library with many of the required features
    % incomplete but functional

% The final implementation of TCP runs without kernel modification, in userspace. 

This project has produced clear and very well documented source code, ideal as a learning tool to supplement traditional teaching techniques. The produced TCP implementation operates well, purely in userspace and without any kernel modifications. Modifying the program or library sources and recompiling takes only a few seconds so is ideal for rapid prototyping of network protocol implementations, such as the provided TCP implementation. The code has been implemented with portability in-mind and therefore does not use any non-standard library functions so should work on many modern Unix implementations with at most minor modifications. Performance of TCP data transmission is more than typical especially for the basic implementation that was used. Not all of the required features ended up in the current library, notably the multiplexed and non-blocking I/O, due to implementation complexity and time constraints although even without them enough functionality is available for the software to be usable in most applications.

\twocolumn


%e TEX root = ../report.tex

\begin{appendices}
\addtocontents{toc}{\protect\setcounter{tocdepth}{1}}
\makeatletter
\addtocontents{toc}{%
  \begingroup
  \let\protect\l@chapter\protect\l@section % chktex 1
  \let\protect\l@section\protect\l@subsection % chktex 1
}
\makeatother

\chapter{Example Configuration Schema}\label{apx:config-schema}

    \section{Examples}
    Given are proposed example schemas for configuration files, in YAML format, to set-up and modify various subsystems within the network stack. All given examples are designed for the current feature-set and capability levels.

        \subsection{Interface Example}\label{interface-example}

        Network interface definitions. Usually at least one of these is required.

        \begin{lstlisting}[language=yaml,gobble=8]
        interfaces:
          # this is the name of the TAP device
          netstack-tap:
            # bridge TAP interface to 'br0'
            type: TAP
            bridged-to: "br0"
            hwaddr: "00:11:22:33:44:55"
            ipaddr:
              - ipv4:    "192.168.128.25/24"
                gateway: "192.168.128.1"
                routes:
                  # this is route is implicit
                  - "192.168.128.1/24"
                  # optional extra routes
                  - address: "10.20.3.16/26"
                    metric:  100
          tun0:
            type: TUN
            address: dhcp
            tunables:
              - respond_icmp: false
        \end{lstlisting}

        \vfil

        \subsection{Logging Example}\label{apx:config-schema-log}

        Logs can be selectively directed to files or streams. Certain log messages can also be suppressed, ideal for pinpointing those pesky bugs hiding amongst the noise.

        \begin{lstlisting}[language=yaml,gobble=8]
        log:
          # killswitch. disables all logging
          # to improve performance
          disabled: true
          output:
            # log output streams
            - file: stdout
              level: "FRAME"
            - file: stderr
              level: "-WARN"
            - file: ~/netstack-debug.log
              level: "DEBUG-"
          # tunables to filter logs
          config:
            - log_hwaddr: false
            - icmp: false
            - ipv4: false
        \end{lstlisting}

        \subsection{Tunable Example}\label{tunable-example}

        Various tunables in the network stack and in protocols can be enabled/disabled and adjusted for optimal performance or feature testing.

        \begin{lstlisting}[language=yaml,gobble=8]
        tunables:
          - respond_icmp: true
          - tcp_windowscaling: false
          - tcp_timewait: 60s
          - tcp_nagle: true
          - tcp_initwindow: 64k
        \end{lstlisting}

        \vfil

        \subsection{ARP Example}\label{arp-example}

        Static ARP entries, such as those for the local network router, can be defined as below. In most cases this is not necessary.

        \begin{lstlisting}[language=yaml,gobble=8]
        arp:
          - ipv4: 192.168.128.1
            # references defined interface
            intf: netstack-tap
            hwtype: ether
            hwaddr: 01:23:45:67:89:AB
        \end{lstlisting}

    \section{File Format}\label{file-format}

        \subsection{Configuration file naming}\label{configuration-file-naming}

        The following names are searched by default by netstack:
        \begin{itemize}[noitemsep,leftmargin=2em]
            \item{\texttt{netstack.conf}}
            \item{\texttt{netstack.yml}}
            \item{\texttt{netstack.yaml}}
        \end{itemize}

        In order from top to bottom, the following directories are searched for a configuration file:
        \begin{itemize}[noitemsep,leftmargin=2em]
            \item{\texttt{/etc/netstack}}
            \item{\texttt{\$XDG\_CONFIG\_HOME}}
            \item{\texttt{\$HOME/.config/netstack}}
        \end{itemize}

        Additionally, many netstack tools (such as \texttt{netd}) should allow overriding the configuration file location from the command line. See the specific documentation for such tools for the appropriate flag.

        \vfil

        \subsection{File structure}\label{file-structure}

        Every netstack configuration file \emph{must} define a schema version with the \texttt{version} top-level tag. Major versions will always be backwards compatible with minor versions. Wherever possible, major versions will also try to be backwards compatible with older major versions however this cannot be guaranteed, for example, in the case of deprecated or removed features.

        \emph{All fixed key names are case-insensitive.}

        \subsection{Top-Level Sections}\label{top-level-sections}

        \begin{itemize}[noitemsep]
            \item{\texttt{version}: A string containing \texttt{x.x} in the form \texttt{major.minor}. \textit{(required)}}
            \item{\texttt{interfaces}: One or more interfaces with configuration for link and internet layer addressing \textit{(recommended)}}
            \item{\texttt{logging}: Log configuration such as filters and log file destinations}
            \item{\texttt{tunables}: Various configurable options such as flags and numeric values}
            \item{\texttt{arp}: Static ARP configuration}
        \end{itemize}

\onecolumn

\chapter{Project Structure}\label{apx:proj-structure}

\section{Overview of the source repository}

\dirtree{%
    .1 \dir{netstack}.
    .2 \dir{tools}\dotfill\DTcomment{A collection of utilities using netstack}.
    .3 \dir{netd}\dotfill\DTcomment{Simple netstack test application}.
    .3 \dir{httpget}\dotfill\DTcomment{A simple webserver query tool}.
    .3 \dir{nshook}\dotfill\DTcomment{netstack init hook library}.
    .3 \exe{netstack-run}\dotfill\DTcomment{netstack injection utility}.
    .2 \dir{tests}.
    .2 \dir{lib}\dotfill\DTcomment{netstack source code}.
    .3 \glob{**.c}.
    .2 \dir{include}.
    .3 netstack.h.
    .3 \dir{netstack}\dotfill\DTcomment{netstack header files}.
    .4 \glob{**.h}.
    .2 .drone.yml\dotfill\DTcomment{CI configuration file}. % chktex 26
    .2 README.md.
    .2 Dockerfile.
    .2 Makefile\dotfill\DTcomment{Primary Makefile. Builds all by default}.
}

\section{Running the example programs}

\begin{enumerate}
    \item Configure static addressing in \texttt{lib/netstack.c} with a valid IP and gateway addresses
    \item Compile everything with the Makefile:\\
        \texttt{\$ CFLAGS=-D\_GNU\_SOURCE make}
    \item Run \texttt{httpget} using \texttt{netstack-run}:\\
        \texttt{\$ tools/netstack-run ./httpget www.cs.bham.ac.uk 80} % chktex 26
    \item Run any program you like using \texttt{netstack-run}:\\
        \texttt{\$ tools/netstack-run <your program here>}
\end{enumerate}

\medskip

\textit{Note:} By default \texttt{netstack-run} will print network stack output to \texttt{stderr}. Append \texttt{2>/dev/null} to the command to suppress the logging output

\twocolumn

\addtocontents{toc}{\endgroup}
\end{appendices}


\printbibliography % chktex 1

%\printglossary[type=\acronymtype]
%\printglossary[type=main]

\end{document}
