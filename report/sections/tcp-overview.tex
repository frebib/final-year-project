%! TEX root = ../report.tex

\section[TCP Overview]{Transmission Control Protocol Overview}
    This overview mostly covers the essentials as defined by the original specification, RFC~793~\cite{rfc793}. Extensions or variations are very useful for improving performance and reliability but absolutely not required for a functioning \gls{TCP} implementation. % chktex 13

    \subsection{Terminology}
        \begin{itemize}
            \item \texttt{SYN}: Synchronise flag\\
                \textit{Set to establish a new connection}
            \item \texttt{FIN}: Finish flag\\
                \textit{Indicate `no more data to send', closing existing connection}
            \item \texttt{ACK}: Acknowledgement flag\\
                \textit{Confirming that data arrived successfully or acknowledging a command}
            \item \texttt{RST}: Reset flag\\
                \textit{Sent in reply to immediately terminate any connection}
            \item Bit field: \textit{Binary data where each individual bit represents a boolean value e.g. 0 or 1}
        \end{itemize}

    \subsection{TCP packet header}
        A \gls{TCP} packet, or segment, contains an initial header followed by the packet payload. The header can be a variable length but is always at least 20 bytes and at most 60 bytes so small payloads are an inefficient use of a segment as there is always a 20+ byte overhead per packet. The header is always a multiple of 32 bits (4 bytes) which is the \textit{word size} for the \gls{TCP} header, and is padded with zeros to fill any remaining space.

        Most of the information carried in the header is metadata for identifying the connection and the data within the packet. Many of the values such as \textit{sequence number} and \textit{window size} correspond to the sender; each peer will have their own values.

        \bigskip
        \begin{figure}[H]
            \centering
            \fontsize{8pt}{10pt}
            \begin{bytefield}[bitheight=2em,bitwidth=0.028\columnwidth]{32}
                \bitheader{0,4,8,10,12,14,16,24,31} \\ % chktex 8
                \bitbox{16}{Source Port} & \bitbox{16}{Destination Port} \\
                \bitbox{32}{Sequence Number} \\
                \bitbox{32}{Acknowledgement Number} \\
                \bitbox{4}{Data offset} & \bitbox{6}{Reserved} &
                \bitbox{1}{\tiny \rotatebox{90}{URG}} & \bitbox{1}{\tiny \rotatebox{90}{ACK}} &
                \bitbox{1}{\tiny \rotatebox{90}{PSH}} & \bitbox{1}{\tiny \rotatebox{90}{RST}} &
                \bitbox{1}{\tiny \rotatebox{90}{SYN}} & \bitbox{1}{\tiny \rotatebox{90}{FIN}} &
                \bitbox{16}{Window} \\
                \bitbox{16}{Checksum} & \bitbox{16}{Urgent Pointer} \\
                \bitbox{24}{Options} & \bitbox{8}{Padding} \\
                \wordbox[lrt]{1}{Payload~\dots} \\
                \skippedwords[0.8em]{} \\
                \wordbox[t]{1/2}{}
            \end{bytefield}
            \vspace*{-1em}
            \caption{\gls{TCP} Packet header}\label{fig:tcphdr}
        \end{figure}

        %In order of appearance, the fields in the header are as follows:
        % \todo{move tcp bits to glossary or appendix or something}
        % \begin{itemize}
        %     \item \textbf{Source Port} \textit{2 bytes} Port from which this packets was dispatched.
        %     \item \textbf{Destination Port} \textit{2 bytes} Port number open to receive on the recipient device.
        %     \item \textbf{Sequence Number} \textit{4 bytes} Sequential count of the first byte in the payload, identifying the sequence position of the segment.
        %     \item \textbf{Acknowledgement number} \textit{4 bytes} Sequential count of successfully received bytes.
        %     \item \textbf{Data offset} \textit{4 bits} Size of the \gls{TCP} header in 32-bit words, as the header is padded to multiples of 4 bytes. This indicates the start of the payload and the end of the header within the packet.
        %     \item \textbf{Flags} \textit{12 bits} A \textit{bit field} containing the connection control flags.
        %     \item \textbf{Window} \textit{2 bytes} Size, in bytes, of the receive window available to fill.
        %     \item \textbf{Checksum} \textit{2 bytes} The \textit{internet checksum}~\cite{internetchecksum} of the entire \gls{TCP} segment and the upper \textit{Internet Protocol} layer. Used to ensure (probable) integrity of the data from corruption.
        %     \item \textbf{Urgent Pointer} \textit{2 bytes} Indicates the offset from the \textit{Sequence number} of the last urgent byte when the \texttt{URG} flag is set.
        %     \item \textbf{Options} \textit{0-40 bytes} Optionally, a series of \gls{TCP} extension options followed by padding to the nearest 4th byte.% chktex 8
        %     \item \textbf{Payload} \textit{0-MSS bytes} Sequenced data for the \gls{TCP} connection. Can be empty. Limited by the \textit{MSS} of the connection.
        % \end{itemize}

    \subsection{Connection-oriented}
        One of the defining features of \gls{TCP} is the \textit{three-way handshake}~(Figure~\ref{fig:threeway}) which every connection begins with.

        \begin{figure}[H]
            \centering
            \begin{tikzpicture}[>=latex,scale=1.2]
                \footnotesize
                \coordinate (TL) at (1.2,3);
                \coordinate (BL) at (1.2,0);
                \coordinate (TR) at (3.4,3);
                \coordinate (BR) at (3.4,0);
                \coordinate (LA) at ($(TL)!.25!(BL)$);
                \coordinate (LB) at ($(TL)!.65!(BL)$);
                \coordinate (LC) at ($(TL)!.70!(BL)$);
                \coordinate (LD) at ($(TL)!.75!(BL)$);
                \coordinate (RA) at ($(TR)!.14!(BR)$);
                \coordinate (RB) at ($(TR)!.40!(BR)$);
                \coordinate (RC) at ($(TR)!.45!(BR)$);
                \coordinate (RD) at ($(TR)!.50!(BR)$);
                \coordinate (RE) at ($(TR)!.90!(BR)$);
                % Lines & labels
                \draw (TL) node[above]{\large Client};
                \draw (TR) node[above]{\large Server};
                \draw[thick] (TL)--(BL) (TR)--(BR);
                % LISTEN
                \draw (RA) node[right]{%
                    \begin{tabular}{l}
                        \verb$LISTEN$\\
                        \textit{passive open}
                    \end{tabular}
                };
                % \texttt{SYN}
                \draw (LA) node[left]{%
                    \begin{tabular}{r}
                        \textit{active open}\\
                        \verb$SYN_SENT$
                    \end{tabular}
                };
                \draw[->] (LA) -- (RB) node[midway,sloped,above]{\verb$SYN$};
                \draw (RC) node[right]{%
                    \begin{tabular}{l}
                        \verb$SYN_RCVD$\
                    \end{tabular}
                };
                % \texttt{SYN}/\texttt{ACK}
                \draw[->] (RD) -- (LB) node[midway,sloped,above]{\verb$SYN/ACK$};
                \draw (LC) node[left]{%
                    \begin{tabular}{r}
                        \verb$ESTABLISHED$\
                    \end{tabular}
                };
                % \texttt{ACK}
                \draw[->] (LD) -- (RE) node[midway,sloped,above]{\verb$ACK$};
                \draw (RE) node[right]{%
                    \begin{tabular}{l}
                        \verb$ESTABLISHED$\
                    \end{tabular}
                };
            \end{tikzpicture}
            \caption{\gls{TCP} Three-way handshake}\label{fig:threeway}
        \end{figure}

        A client requests a new connection by sending an initial packet with the \texttt{SYN} bit set, as well as an initial random \textit{sequence number}. When a packet marked with \texttt{SYN} is received at the server, a new connection is being established and the server replies setting both \texttt{SYN} and \texttt{ACK} bits in the packet. This reply does one of two things:
        \begin{enumerate}
            \item The mutual \texttt{SYN} flag indicates the connection is being accepted along with a separate random sequence number.
            \item The \texttt{ACK} flag being set along with the \textit{acknowledgement number} indicates it understood the initial \textit{sequence number}.
        \end{enumerate}
        The sequence number from the initial flag is reused, incremented by one and sent back as the acknowledgement number in the \texttt{SYN}/\texttt{ACK} packet.

        Finally, the client replies with the sequence number sent by the server, incremented by one, which completes the three-way handshake. At this point the connection is set up and ready to be used to transmit data.

        When the connection is finished with, both endpoints need to mutually close the connection, in a very similar fashion to the \texttt{SYN}, \texttt{SYN}/\texttt{ACK}, \texttt{ACK} pattern.
        Either side of the connection can initiate a \texttt{FIN} at any point, indicating there is no more data to send to which the peer must respond with an \texttt{ACK} of the final sequence number incremented by one to confirm the \texttt{FIN} packet was received.

        \texttt{FIN} packets can behave slightly differently to \texttt{SYN} packets as there is no requirement to send a \texttt{FIN} packet immediately after receiving one. In fact, a connection can remain \textit{half-closed} for some time whilst one side finishes transmitting data. This is a common pattern for some software applications as only \gls{simplex} communication is required. It is also acceptable to close a connection quickly with \texttt{FIN}, \texttt{FIN}/\texttt{ACK}, \texttt{ACK}.

        If at any point a \texttt{SYN}, \texttt{FIN} or other packet has not been acknowledged by the peer within a certain time frame, the packet will be resent under the assumption it never arrived; this is part of the reliability of \gls{TCP} (See Section~\ref{sec:reliable}).

        Following this pattern ensures both peers are in agreement of the state of the connection at all times. If at any point an invalid connection request is sent, to a closed port for example, it is instantly replied to with an \texttt{RST} packet indicating that the client should immediately discard the connection as it is invalid.

    \subsection{Ordered}
        \textit{Sequence numbers} are used to track each individual byte sent over the connection. For each byte successfully received by the peer, the acknowledgement number is incremented to that value. Each packet contains a sequence number for the first byte of the payload in that segment indicating the position of the data in the continuous stream. If a packet arrives out of order, it can be kept aside until it is required or dropped and it will be retransmitted. Data that has been sent but not acknowledged is assumed to be lost so, after a timeout, it is retransmitted.

        For the most part, especially with small local networks or across short distances, packets will arrive in order and complete so little retransmission is required, just some reordering and payload reassembly.

        There are various methods of tracking missing packets, storing and reordering packets, indicating that packets are lost to reduce retransmission latency and so on that are used in \gls{TCP} implementations which can improve performance and throughput in many cases, however, they are not required for basic \gls{TCP} functionality.

    \subsection{Reliable}\label{sec:reliable}
        Acknowledging data is the first method used by \gls{TCP} to ensure that all data arrives intact. The sender tracks the sequence number of the last acknowledgement it received. It also buffers the transmitted but unacknowledged data in the case of it having to be retransmitted. As data is received at the client, acknowledgements are sent in reply confirming the data was received.

        Complementing that, a primitive checksum known as the \textit{Internet Checksum} is used to verify integrity of the data. The \textit{Internet Checksum} is defined as ``the 16 bit one's complement of the one's complement sum of all 16 bit words in the header and text''~\cite[3.1]{rfc793}.

        By modern standards it is a very poor checksum design, however it is very fast to compute and is still sufficient to ensure that no serious data corruption has occurred. In most cases, even if the checksum is still valid, corruption caused to the header could be detected and the packet will be discarded likely due to out-of-band sequence numbers or similar. As the checksum is simple in nature, it can only provide basic data integrity and does not protect against malicious attacks in the way that cryptographic hashes do.
        
        \subsection{Extensible}
        \gls{TCP} has a vast array of extensions, specified in the `Option' section at the end of the packet header, that are used to fix many issues, improve performance and add extra functionality. This section covers the commonly used options of note, but there are many more that are widely used and some still in development.

        \subsubsection{Window Scaling}\label{sec:wscale}
            At the advent of \gls{TCP}, networks were comparatively slow at only a few Megabits per second, however, technology has advanced and now links are available in excess of 100 Gigabits per second. The default \textit{Window size} defined in \gls{TCP} is only 64KiB which at 100Gbps can be filled in just 5 microseconds. Small receive windows can slow down transmission dramatically as the buffer has to be cleared before it can be populated again by the sender. To leverage these high transfer speeds, Window Scaling~\cite[2]{rfc1323} was introduced to increase the size of the default receive window.

            Window scaling must be negotiated during the initial handshake of the connection; a client requests \texttt{wscale} in the first \texttt{SYN} packet to which the peer replies with a \texttt{wscale} option in the returning \texttt{SYN}/\texttt{ACK} segment, confirming the choice to use scaling for the remainder of the connection. Should no reply to the \texttt{wscale} option be sent, it is assumed the peer does not support window scaling and therefore the connection must operate without scaling.

            The option specifies a scale value between 0 and 14 which should be applied as a \textit{bitwise left shift} to the window size allowing a maximum receive window size of 1GiB ($65536 << 14$). It is completely valid to negotiate a `zero' shift in one direction to disable scaling for one client.

        \subsubsection{Selective Acknowledgement}\label{sec:sack}
            Acknowledgements are part of the reason \gls{TCP} is so useful, but in particularly poor network conditions the performance of \gls{TCP} can suffer from excessive retransmissions, driving latency up and throughput down. Often in lossy transfers only a small number of packets will be lost but \gls{TCP} defines that data should be acknowledged until that point and all data afterwards will be retransmitted. It is often the case that most or all data after a missing packet will have arrived successfully but would be unnecessarily retransmitted.

            Selective \texttt{ACK} aims to reduce and abolish retransmissions for data that has already arrived by indicating to the sender which blocks have or have not arrived. It is an option appended specifically to \texttt{ACK} packets to give more detail about which byte ranges should be retransmitted. The option has a variable payload length to allow inclusion of the left and right boundaries of each block that is being acknowledged.

        \subsubsection{\citeauthor{rfc896}'s Algorithm}
            \citeauthor*{rfc896} proposed in \citeyear{rfc896}~\cite{rfc896} the concept of reducing the amount of small packets being transmitted by buffering outgoing data for a small period of time before sending it. This is today known as \textit{Nagle's Algorithm} which is defined to address the ``small packet problem'' by not transmitting new data whilst there is still unacknowledged data in transmission. Sent data is queued indefinitely in the transmit buffer until an acknowledgement is received or until the buffer contains enough data to fill at least one segment.

            The segment size in this case is the \textit{Maximum Segment Size} calculated by the lowest \gls{MTU} in the route between the two communicating nodes, minus the header sizes. The MSS tends to be 1460 bytes, or less, assuming the standard Ethernet (IEEE 802.3) \gls{MTU} of 1500 is used.

        \subsubsection{TCP Corking}
            Linux implements a feature typically referred to as \texttt{\gls{TCP}\_CORK} which is similar in nature to the operation of Nagle's Algorithm. The ultimate goal of corking is the same as Nagle, but is achieved in a different way. Corking will always buffer data, even after the reception of an acknowledgement, until there is enough to fill a segment or until the timeout has passed.  Linux uses a default corking timeout of 200ms.

        %\subsubsection{Congestion Control}
            %One of the final main features of \gls{TCP} is its ability to survive and recover from network congestion with the use of complex control algorithms. Congestion control actually refers to a set of tools that a \gls{TCP} should have, discussed in detail in the following sections:

        \subsubsection{Slow Start}
            A connection between two nodes is limited by the speed of the slowest intermediary node, such as a router, network interface or switch. To avoid overwhelming any of these devices resulting in packet loss, a \gls{TCP} should assume the worst-case and start transmitting slowly. Congestion controlled \gls{TCP} implementations maintain \textit{congestion window} which is the representation to the sender of the state of congestion in the network between the connected peers.

            Slow start initialises a small congestion window, usually a small multiple of the \textit{maximum segment size}. Each time data is transmitted, the sender only releases at most the amount specified by the congestion window. For each acknowledgement received for all transmitted data, the congestion window is increased, often by a factor of two. Each subsequent transmission will be larger than the previous as the congestion window increases. When a lost segment is detected, slow start will stop and the standard linear growth Congestion Avoidance algorithm will kick in.

        \subsubsection{Congestion Avoidance}
            There have been many varying implementations of Congestion Avoidance algorithms with ranging feature sets for different scenarios. Some offer fast recovery, smaller loss or improved throughput. The aim of these algorithms is to continue a stable transmission between the two hosts whilst preventing subsequent packet loss by over-sending and overloading the network.

            Popular algorithms for this are: Vegas, BIC, CUBIC, Westwood, Reno, Tahoe and New Reno. Typically `New Reno' is used in Unix systems today as it provides the best performance and stability with minimal changes required to either sender or receiver.
