%! TEX root = ../report.tex

\chapter{Specification}\label{chp:spec}
    % How you analysed the problem
    % Give an appropriate specification of the solution

    % Define specific behavior or functions
    % what a software system should do
    \section{Functional Requirements}
    \begin{itemize}
        \item This project must produce a TCP/IP implementation that runs completely in Linux userspace with no kernel modification.
        \item This project must follow the original RFC 793~\cite{rfc793} specification for TCP.
        \item This project could implement newer TCP extensions or improvements.
        \item The implementation should be portable across Linux and Unix-like systems.
        \item This project must provide a network stack as a library such that it can be easily implemented in other software.
        \item This project should provide functionality to inject the network stack into existing applications.
        \item Customisability and plugability could be implemented to allow expansion and/or modification of new and existing protocols.
        \item Adapting the library or binaries should be flexible and should provide various modes of operation to suit a wide range of systems.
    \end{itemize}

    % A requirement that specifies criteria that can be used to judge the operation of a system, rather than specific behaviors.
    % constraints on how the system should do so
    % elaborates a performance characteristic
    \section{Non-functional Requirements}
    \begin{itemize}
        \item The code should be sufficiently commented such that it can be easily understood, modified and learned from.
        \item Performance of data transfers within the TCP subsystem should be able to sustain at least 100 megabit/second average throughput to stay relevant in modern network scenarios.
    \end{itemize}

    % What the user expects the software to be able to do
    \section{User Requirements}
    To be able to quantify requirements for a user, first the different types of users of the application must be defined.

        \subsection{Student}
        Using the project as a learning tool demands in-depth documentation and clear, commented code. Users will need to be able to follow through the code and cross-reference it with the logs as it runs. Running the software against existing programs and linking with new programs should be as simple as using any other standard dynamic C library.

        \subsection{Researcher}
        Customisability of internal protocols should be straight-forward with minimal code refactoring required to perform the desired changes. The possibility of adding new or modified protocols into the existing stack should require only the addition of only a few lines of code without having to refactor the layering to accommodate new protocol designs.

