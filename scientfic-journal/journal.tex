\documentclass[10pt,a4paper,british]{article}
\usepackage[margin=7.5em]{geometry}
\usepackage[utf8]{inputenc}
\usepackage{babel}
\usepackage{csquotes}
\usepackage{isodate}

\usepackage[backend=biber]{biblatex}
\usepackage[hidelinks]{hyperref}
\addbibresource{references.bib}

\begin{document}

\begin{abstract}
This is an abstract
\end{abstract}

\tableofcontents

\section{Introduction}

\section{Literature Review}
% Difficult to find papers about TCP implementations as there aren't many
% Focussed on userspace implementations as issues found are likely to be similar
\subsection{An Experimental User Level Implementation of TCP~\cite{braun:inria-00074040}}
\subsection{Design and Implementation of the lwIP TCP/IP Stack~\cite{lwip}}
\subsection{mTCP: a Highly Scalable User-level TCP Stack for Multicore Systems} %chktex 13
\cite{jeong2014mtcp} % chktex 2

\section{TCP Overview}

\section{Implementation Detail}

\section{Implementation Issues}
\subsection{Hijacking the BSD \texttt{socket()} API calls} % chktex 36
% 

\subsection{Frames larger than the interface MTU}
% Caused by 'segmentation offloading'
% Affects TCP checksums on packets greater than MSS/MTU

\subsection{Fighting the Linux TCP/IP stack RST packets}
% Using iptables or libiptc to drop RST on specific ports
%  - Port can't be used by the kernel and could clash?
% Opening the socket to accept the connection
%  - Eventually socket will timeout
%  - Doesn't block any packets from actually getting out
% Kernel module/extension to prevent RST being generated
%  - Could be quicker but defeats isn't truly userspace

\subsection{Efficiently sending and receiving packets through the kernel~\cite{tpacket}}
% "it requires one system call to capture each packet, it requires two if you want to get packet's timestamp"
% Use a zero-copy interface like Linux tpacket_v{1,2,3}

\printbibliography{}
\end{document}
